\documentclass{article}

\usepackage[french]{babel}
\usepackage[utf8]{inputenc}
\usepackage[T1]{fontenc}
\usepackage{listings}

\title{Tutoriel Rails}
\author{L3 Informatique---Sécurité Informatique}
\date{Université Paul Sabatier Toulouse 3}

\lstset{language=bash}
\begin{document}

\maketitle

\section{Déploiement}

\begin{description}
\item[Ajout de librairies:] ~\\
  \begin{enumerate}
  \item éditer le fichier Gemfile pour ajouter la librairie (la ligne à ajouter est de la forme:
    \begin{lstlisting}
      gem 'librairie'
    \end{lstlisting}
  \item sauvegarder, et télécharger la librairie:
    \begin{lstlisting}
      bundle install
    \end{lstlisting}
  \end{enumerate}
\item[Lancement du serveur Web:]  ~\\
    \begin{lstlisting}
      rails s
    \end{lstlisting}
\item[Lancement d'une console:]  ~\\
    \begin{lstlisting}
      rails c
    \end{lstlisting}
\end{description}


\section{Interaction avec la base de données}

La base de données actuelle est décrite dans le fichier
\texttt{db/schema.rb}.  Ce fichier ne doit pas être édité
manuellement. Lorsqu'on veut changer cette base, il faut créer une
nouvelle \emph{migration} qui décrit les changements à faire, et
ensuite appliquer cette migration. On peut revenir en arrière avec:
\begin{lstlisting}
  rake db:rollback
\end{lstlisting}

Les principales commandes sont:
\begin{description}
\item[Création:] ~\\
  \begin{lstlisting}
    rake db:migrate
  \end{lstlisting}
\item[Changement dans les tables existantes:] ~\\
  \begin{enumerate}
  \item Un changement s'appelle une \emph{migration}. On crée une
    nouvelle migration vide avec:
    \begin{lstlisting}
      rails g migration nom_migration
    \end{lstlisting}
    Dans certains cas (ajout d'une colonne), la migration peut être générée depuis
    la ligne de commande. Par exemple, pour ajouter une colonne \texttt{age} de type \texttt{int}
    à la table \texttt{users}, on peut utiliser:
    \begin{lstlisting}
      rails g migration add_age_to_users age:int
    \end{lstlisting}
  \item Mettre à jour la base de données:
    \begin{lstlisting}
      rake db:migrate
    \end{lstlisting}
    \end{enumerate}
\end{description}
Les migrations permettent aussi de créer de nouvelles tables, mais en
général ces tables sont liées à des \textbf{modèles}, et seront créées automatiquement lors de la création du modèle. Il faudra cependant toujours faire la mise à jour avec:
    \begin{lstlisting}
      rake db:migrate
    \end{lstlisting}
 

\section{Modèles et scaffold}

\subsection{Modèles}

Un modèle est défini par une classe. Par convention, le modèle
\texttt{User} est défini par le fichier \texttt{user.rb} dans le
répertoire \texttt{app/models}. 

\begin{description}
\item[Création d'un nouveau modèle:] La commande:\\
  \begin{lstlisting}
    rails g model shoe  user:references size:int:index couleur:string
  \end{lstlisting}
  crée un nouveau modèle \texttt{Shoe}. Dans la base de données,
  chaque enregistrement contiendra une clef étrangère
  (\texttt{references}) venant de la table \texttt{users}, une valeur
  \texttt{size} de type \texttt{int} avec un index sur cette colonne
  (pour pouvoir récupérer rapidement les chaussures d'une taille
  donnée), et enfin une chaîne de caractères qui en décrit la couleur.
  Plus précisément, cette commande a créé la migration qui va créer cette table,
  et il faut maintenant appliquer cette migration:
  \begin{lstlisting}
    rake db:migrate
  \end{lstlisting}
\item[Utilisation des relations de la BD dans l'application:] 
  \begin{itemize}
  \item si l'enregistrement contient une référence, on utilise une
    relation \texttt{belongs\_to} pour accéder à l'enregistrement
    référencé:
    \begin{lstlisting}[language=ruby]
class Shoe < ActiveRecord::Base
  belongs_to :user
end
    \end{lstlisting}
  \item Si un utilisateur peut avoir une seule chaussure, on écrira
    dans \texttt{app/models/user.rb}:
    \begin{lstlisting}[language=ruby]
class User < ActiveRecord::Base
  has_one :shoe
end
    \end{lstlisting}
et si un utilisateur peut en avoir plusieurs, on écrira plutôt:
    \begin{lstlisting}[language=ruby]
class User < ActiveRecord::Base
  has_many :shoes
end
    \end{lstlisting}
    (notez le passage du singulier au pluriel).
  \end{itemize}
\item[Création d'un enregistrement:] 
  \begin{enumerate}
  \item ouvrir la console:
    \begin{lstlisting}
      rails c
    \end{lstlisting}
  \item créer la nouvelle instance:
    \begin{lstlisting}[language=ruby]
chaussure=Shoe.new({user: User.first, size: 36, couleur: 'rouge'})
    \end{lstlisting} 
    (User.first est le premier enregistrement trouvé dans la table
    \texttt{users})
  \item sauvegarder cette instance:
    \begin{lstlisting}[language=ruby]
chaussure.save
    \end{lstlisting} 
    on aussi peut le faire en une seule étape:
    \begin{lstlisting}[language=ruby]
      Shoe.create({user: User.first, size: 36, couleur: 'rouge'})
    \end{lstlisting}
  \end{enumerate}
\item[Retrouver des modèles:] On fait une recherche sur les modèles en
  utilisant \texttt{where}. Par exemple (console rails):
  \begin{lstlisting}[language=ruby]
    taille36=Shoes.where({size: 36}) 
  \end{lstlisting}
  (\texttt{where} rend un tableau possiblement vide). 
\end{description}

\subsection{Scaffold}

Dans l'énoncé du TP, les tables correspondant uniquement à des modèles
sont en rouge. Celles qui sont en bleues peuvent être manipulées
(ajout/suppression d'enregistrement, visualisation, édition) à travers
l'interface Web.
\begin{description}
\item[Contrôleur:] un contrôleur contient des méthodes à exécuter. Par
  défaut, ce sont les méthodes ReST: show, new (formulaire de
  création), create (création), edit (formulaire de modification),
  update (modification), index, et destroy. Par convention, pour le
  modèle \texttt{Shoe}, le contrôleur est
  \texttt{app/controllers/shoes\_controller.rb}.
\item[Routage:] le fichier \texttt{config/routes.rb} contient la
  liste des actions à faire lors de la réception d'une requête HTTP. Par défaut,
la déclaration:
\begin{lstlisting}[language=ruby]
 resources :shoes
\end{lstlisting}
va créer 7 couples (url+verbe http) qui seront traduites en appels de
méthode dans le contrôleur (par exemple, le couple \(('/shoes',get)\)
va conduire à appeler la méthode \texttt{index} du contrôleur
\texttt{shoes\_controller.rb}.
\item[Vues:] les méthodes du contrôleur assignent des variables
  globales (dont le nom commence par @) qui sont ensuite utilisées
  pour construire la réponse. Pour chaque méthode par défaut, la
  réponse est construite à partir d'un des fichiers du répertoire
  \texttt{app/views/shoes/}. Les fichiers\texttt{.html.erb} mêlent de l'html
  et du ruby de 2 manières:
  \begin{itemize}
  \item lorsque le résultat de l'évaluation du code ruby doit être inséré, on utilise la balise
    \verb+<%=+;
  \item sinon, on utilise la balise \verb+<%+;
  \end{itemize}
\end{description}

Tous ces fichiers sont générés automatiquement si, au lieu d'utiliser:
\begin{lstlisting}
  rails g model shoe  user:references size:int:index couleur:string
\end{lstlisting}
on utilise:
\begin{lstlisting}
  rails g scaffold shoe  user:references size:int:index couleur:string
\end{lstlisting}
Notez que dans ce cas, comme on modifie la configuration (le fichier
\texttt{config/routes.rb}), il faut relancer le serveur.
\end{document}
